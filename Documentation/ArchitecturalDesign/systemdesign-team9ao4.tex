\documentclass[12pt]{article}

% Imported Packages
%------------------------------------------------------------------------------
\usepackage{placeins}
\usepackage{amssymb}
\usepackage{amstext}
\usepackage{amsthm}
\usepackage{amsmath}
\usepackage{enumerate}
\usepackage{fancyhdr}
\usepackage[margin=1in]{geometry}
\usepackage{graphicx}
\usepackage{extarrows}
\usepackage{setspace}
\usepackage[utf8]{inputenc}
%------------------------------------------------------------------------------

% Header and Footer
%------------------------------------------------------------------------------
\pagestyle{plain}  
\renewcommand\headrulewidth{0.4pt}                                      
\renewcommand\footrulewidth{0.4pt}                                    
%------------------------------------------------------------------------------

% Title Details
%------------------------------------------------------------------------------
\title{Large System Design\\
	\large Carspot for SE 3A04, Tutorial 2}
    \author{
         Yasaswi Gopalkrishnan\\ \newline
         \and
         Sharon Platkin \\ \newline
         \and 
         Abhijit Singh Dhoat\\ \newline
         \and
         Joseph Cole Huot\\ \newline
         \and
         David Eric Hemms\\ \newline
         \and 
         Yuchen Liu\\ \newline
    }   
    \date{Monday March 7th, 2016}                             
%------------------------------------------------------------------------------

% Document
%------------------------------------------------------------------------------
\begin{document}

\maketitle
\newpage
\tableofcontents
\listoftables
\newpage	

\section{Introduction}
\label{sec:introduction}
% Begin Section

This section should provide an brief overview of the entire document.

\subsection{Purpose}
\label{sub:purpose}
% Begin SubSection
\begin{enumerate}[a)]
	\item Delineate the purpose of the document
	\item Specify the intended audience for the document
\end{enumerate}
% End SubSection

\subsection{System Description}
\label{sub:system_description}
% Begin SubSection
\begin{enumerate}[a)]
	\item Give a brief description of the system. This could be a paragraph or two to give some context to this document.
\end{enumerate}
% End SubSection

\subsection{Overview}
\label{sub:overview}
% Begin SubSection
\begin{enumerate}[a)]
	\item Describe what the rest of the document contains 
	\item Explain how the document is organised
\end{enumerate}
% End SubSection

% End Section

\section{Use Case Diagram}
\label{sec:use_case_diagram}
% Begin Section
This section should provide a use case diagram for your application. 
\begin{enumerate}[a)]
	\item Each use case appearing in the diagram should be accompanied by a text description. 
\end{enumerate}
% End Section

\section{Analysis Class Diagram}
\label{sec:analysis_class_diagram}
% Begin Section
This section should provide an analysis class diagram for your application.
% End Section


\section{Architectural Design}
\label{sec:architectural_design}
% Begin Section
This section should provide an overview of the overall architectural design of your application. You overall architecture should show the division of the system into subsystems with high cohesion and low coupling.

\subsection{System Architecture}
\label{sub:system_architecture}
% Begin SubSection
\begin{enumerate}[a)]
	\item The system is based on a blackboard architecture. There are four separate experts who can provide information independently using their expertise. Each expert identifies a different car property. A car search uses the information provided by the experts to search the car database, finding cars which have the identified properties.
	\item This architecture structure works well for this system because it is a knowledge based system. Each expert can provide information which is then used to make a decision. Experts can also be added or removed very easily which gives the system flexibility. The experts are independent of one another, giving the system low coupling. An individual expert has one property which it will identify, giving high cohesion.
	\item Structural architecture diagram of the system:
	\includegraphics{Structural.png}
\end{enumerate}
% End SubSection

\subsection{Subsystems}
\label{sub:subsystems}
% Begin SubSection
\begin{enumerate}[a)]
	\item {\textbf{Blackboard Subsystems}}
	\item Car Search:
	\item This subsystem uses car properties provided by the experts to find car models in the database which have the provided properties.
	\item {\textbf{Knowledge Source Subsystems}}
	\item Car Type:
	\item An expert which identifies the type of car (Sedan, SUV, Minivan, etc).
	\item Company Logo:
	\item An expert which identifies the company that made the car based on their logo.
	\item Car Origin:
	\item An expert which identifies the origin of the car (North American, European, etc).
	\item Passenger Caoacity:
	\item An expert which identifies the number of passengers the car can hold.
	\item Database:
	\item A database containing car models and their properties. The database can be searched to find models which fit certain criteria.
	\item {\textbf{Controller Subsystem}}
	\item Control:
	\item This subsystem can initiate a car search and supervise the overall identification process.
\end{enumerate}
% End SubSection

% End Section
	
\section{Class Responsibility Collaboration (CRC) Cards}
\label{sec:class_responsibility_collaboration_crc_cards}
% Begin Section

	\begin{table}[ht]
		\centering
		\begin{tabular}{|p{5cm}|p{5cm}|}
		\hline 
		 \multicolumn{2}{|l|}{\textbf{Class Name:} CarDB} \\
		\hline
		\textbf{Responsibility:} & \textbf{Collaborators:} \\
		\hline
		Contain a listing of all car models and their attributes & -\\
		\hline
		Allow insertion and deletion of entries & -\\
		\hline
		Allow editing of entries & - \\
		\hline
		Provide information to CarSearchController & CarSearchController \\
		\hline
		\end{tabular}
	\end{table}
	
	\begin{table}[ht]
		\centering
		\begin{tabular}{|p{5cm}|p{5cm}|}
		\hline 
		 \multicolumn{2}{|l|}{\textbf{Class Name:} FeedbackStorage} \\
		\hline
		\textbf{Responsibility:} & \textbf{Collaborators:} \\
		\hline
		Contain a list of all feedback forms completed by users with anonymity, stored in a file & -\\
		\hline
	    Receive feedback from feedback form for storage & FeedbackForm\\
		\hline
		\end{tabular}
	\end{table}
	
	\begin{table}[ht]
		\centering
		\begin{tabular}{|p{5cm}|p{5cm}|}
		\hline 
		 \multicolumn{2}{|l|}{\textbf{Class Name:} FeedbackForm} \\
		\hline
		\textbf{Responsibility:} & \textbf{Collaborators:} \\
		\hline
	    Allow user to enter feedback about the application & -\\
		\hline
		\end{tabular}
	\end{table}
	
	\begin{table}[ht]
		\centering
		\begin{tabular}{|p{5cm}|p{5cm}|}
		\hline 
		 \multicolumn{2}{|l|}{\textbf{Class Name:} CarSearchController} \\
		\hline
		\textbf{Responsibility:} & \textbf{Collaborators:} \\
		\hline
		Contains algorithm to identify a car given some attributes & -\\
		\hline
	    Extract information from the SearchForm and compile it into a search query & SearchForm\\
	    \hline
	    Send result of search to SearchResult for display and verification & SearchResult\\
	    \hline
	    Query car database and experts as part of search algorithm to identify the car & CarDB, Expert\\
	    \hline
	    Control experts to be used in identification based on attributes given & ExpertPicker\\
		\hline
		\end{tabular}
	\end{table}

	\begin{table}[ht]
		\centering
		\begin{tabular}{|p{5cm}|p{5cm}|}
		\hline 
		 \multicolumn{2}{|l|}{\textbf{Class Name:} SearchResult} \\
		\hline
		\textbf{Responsibility:} & \textbf{Collaborators:} \\
		\hline
		Receive search result and send it to the forum to be displayed  & Forum,CarSearchController\\
		\hline
		Once a car identification is confirmed, result sent to search history & SearchHistory\\
		\hline
		Send result for verification before sending to search history & ResultVerifier\\
		\hline
		\end{tabular}
	\end{table}

	\begin{table}[ht]
		\centering
		\begin{tabular}{|p{5cm}|p{5cm}|}
		\hline 
		 \multicolumn{2}{|l|}{\textbf{Class Name:} ExpertPicker} \\
		\hline
		\textbf{Responsibility:} & \textbf{Collaborators:} \\
		\hline
		Control which experts will be used to identify the car based on attributes that are inputted & Expert\\
		\hline
		Set experts to "passive" or "active" for identification process & Expert\\
		\hline
		\end{tabular}
	\end{table}
	
	\begin{table}[ht]
		\centering
		\begin{tabular}{|p{5cm}|p{5cm}|}
			\hline 
			\multicolumn{2}{|l|}{\textbf{Class Name:} HelpPage} \\
			\hline
			\textbf{Responsibility:} & \textbf{Collaborators:} \\
			\hline
			Provide information about the application, and how to use it & -\\
			\hline
		\end{tabular}
	\end{table}

	\begin{table}[ht]
		\centering
		\begin{tabular}{|p{5cm}|p{5cm}|}
			\hline 
			\multicolumn{2}{|l|}{\textbf{Class Name:} Forum} \\
			\hline
			\textbf{Responsibility:} & \textbf{Collaborators:} \\
			\hline
			Central hub of application to allow navigation to various pages & SearchForm, SearchHistory, HelpPage, FeedbackForm\\
			\hline
			Display result of car identification & SearchResult\\
			\hline
		\end{tabular}
	\end{table}
	
	\begin{table}[ht]
		\centering
		\begin{tabular}{|p{5cm}|p{5cm}|}
			\hline 
			\multicolumn{2}{|l|}{\textbf{Class Name:} SearchForm} \\
			\hline
			\textbf{Responsibility:} & \textbf{Collaborators:} \\
			\hline
			Allow user to input characteristics of the car they want to identify & -\\
			\hline
			Send inputted attributes to car identification algorithm & CarSearchController\\
			\hline
		\end{tabular}
	\end{table}

	\begin{table}[ht]
		\centering
		\begin{tabular}{|p{5cm}|p{5cm}|}
			\hline 
			\multicolumn{2}{|l|}{\textbf{Class Name:} SearchHistory} \\
			\hline
			\textbf{Responsibility:} & \textbf{Collaborators:} \\
			\hline
			Store previous five confirmed identification results & -\\
			\hline
			When a new result enters the history, pushes out fifth most recent confirmed identification & -\\
			\hline
		\end{tabular}
	\end{table}

	\begin{table}[ht]
		\centering
		\begin{tabular}{|p{5cm}|p{5cm}|}
			\hline 
			\multicolumn{2}{|l|}{\textbf{Class Name:} DealershipLocator} \\
			\hline
			\textbf{Responsibility:} & \textbf{Collaborators:} \\
			\hline
			Interface with Google Maps API to locate dealerships that sell a specific car from the search history & SearchHistory\\
			\hline
		\end{tabular}
	\end{table}

	\begin{table}[ht]
		\centering
		\begin{tabular}{|p{5cm}|p{5cm}|}
			\hline 
			\multicolumn{2}{|l|}{\textbf{Class Name:} SecurityController} \\
			\hline
			\textbf{Responsibility:} & \textbf{Collaborators:} \\
			\hline
			Contains encryption and decryption mechanisms for transmitted messages & -\\
			\hline
			Decrypt search result once it arrives at the forum & Forum\\
			\hline
			Encrypt the search result before sending it to the forum & SearchResult\\
			\hline
		\end{tabular}
	\end{table}

	\begin{table}[ht]
		\centering
		\begin{tabular}{|p{5cm}|p{5cm}|}
			\hline 
			\multicolumn{2}{|l|}{\textbf{Class Name:} ResultVerifier} \\
			\hline
			\textbf{Responsibility:} & \textbf{Collaborators:} \\
			\hline
			Provide the user with the ability to confirm or deny the identified car result & -\\
			\hline
			Restart car identification if identified car is incorrect & CarSearchController\\
			\hline
			Restart search form if the identified car is incorrect three times & CarSearchController, SearchForm\\
			\hline
		\end{tabular}
	\end{table}

	\begin{table}[ht]
		\centering
		\begin{tabular}{|p{5cm}|p{5cm}|}
			\hline 
			\multicolumn{2}{|l|}{\textbf{Class Name:} Expert} \\
			\hline
			\textbf{Responsibility:} & \textbf{Collaborators:} \\
			\hline
			Know potential car identifications given certain attribute combinations in respective domain of expertise & -\\
			\hline
			Provide expertise to identify a car given some attributes of its domain & CarSearchController\\
			\hline
			Provide functionality to be set as "active" or "passive" when trying to identify a car & ExpertPicker\\
			\hline
		\end{tabular}
	\end{table}
% End Section

\FloatBarrier
\appendix
\section{Division of Labour}
\label{sec:division_of_labour}
% Begin Section
\begin{table}[ht]
	\centering
	\begin{tabular}{|p{5cm}|p{5cm}|}
		\hline 
		\textbf{Team Member:} & \textbf{Sections Completed:}\\
		\hline
		Abhijit & Section 1, 4\\
		\hline
		Cole & Section 3, 4, Reviewed and Reworked Business Events\\
		\hline
		David & Section 3, 5, Reviewed and Reworked Business Events\\
		\hline
		Sharon & Section 2, 3, Reviewed and Reworked Business Events\\
		\hline
		Yash & Section 3, 5, Reviewed and Reworked Business Events\\
		\hline
		Yuchen & Section 4 Reviewed and Reworked Business Events\\
		\hline
	\end{tabular}
\end{table}
% End Section

\newpage
\section*{IMPORTANT NOTES}
\begin{itemize}
%	\item You do \underline{NOT} need to provide a text explanation of each diagram; the diagram should speak for itself
	\item Please document any non-standard notations that you may have used
	\begin{itemize}
		\item \emph{Rule of Thumb}: if you feel there is any doubt surrounding the meaning of your notations, document them
	\end{itemize}
	\item Some diagrams may be difficult to fit into one page
	\begin{itemize}
		\item It is OK if the text is small but please ensure that it is readable when printed
		\item If you need to break a diagram onto multiple pages, please adopt a system of doing so and thoroughly explain how it can be reconnected from one page to the next; if you are unsure about this, please ask about it
	\end{itemize}
	\item Please submit the latest version of Deliverable 1 with Deliverable 2
	\begin{itemize}
		\item It does not have to be a freshly printed version; the latest marked version is OK
	\end{itemize}
	\item If you do \underline{NOT} have a Division of Labour sheet, your deliverable will \underline{NOT} be marked
\end{itemize}


\end{document}
%------------------------------------------------------------------------------